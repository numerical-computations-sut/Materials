گزاره‌های زیر را اثبات کنید:
\begin{enumerate}
	\item
	در صورتی که 
	$f[x_0,x_1,x_2,...,x_n]$
	تفاضلات تابع دلخواه 
	$f$
	در نقاط 
	$x_0$
	تا
	$x_n$
	باشد،
	
	
	$f[x_0,x_1,...,x_n] = \sum_{i=0}^{n}\frac{f(x_i)}{ \displaystyle  \prod_{j=0 \atop j\neq i}^{n}(x_i - x_j)}$
	
	\item
	فرض کنید 
	$f$
	در بازه‌ی شامل 
	$x_0,x_1,...x_n$،
	$n$
	بار مشتق‌پذیر است. در این صورت به ازای 
	$\xi$
	در بازه‌ی شامل نقاط
	$x_0,x_1,...,x_n$
	ثابت کنید:
	
	\begin{center}
		$f[x_0,...,x_n] = \frac{f^{(n)}(\xi)}{n!}$
	\end{center}
\end{enumerate}

\textcolor{blue}{حل
\begin{enumerate}
    \item 
    می‌دانیم در روش
    \LR{Divided Difference}:
    \begin{align} \label{eq:1}
        \text{چندجمله‌ای درونیابی} = f(x_0) + (x - x_0) f[x_0, x_1] + \dots + (x - x_0) \dots (x - x_n) f[x_0, \dots, x_n]
    \end{align}
    پس عبارت فوق چندجمله‌ای از درجه
    $n$
    است که ضریب 
    $x^n$
    در آن برابر 
    $f[x_0, \dots, x_n]$
    است. 
    \\
    اما اگر سوال را با لاگرانژ حل می‌کردیم:
    \begin{align*}
    \text{درونیابی شده
    $f$} = \sum_{i = 0}^{n} L_i (x) f(x_i) = 2 \stackrel{L_i(x) = \prod_{j = 0 \neq i}^{n} \frac{(x - x_j)}{(x_i - x_j)}}{=} \sum_{i = 0}^{n} (\prod_{j = 0 \neq i}^{n} \frac{(x- x_j)}{(x_i - x_j)}) f(x_i)
    \end{align*}
    با توجه به این که پشت
    $x$
    ها عددی نیست، ضریب 
    $x^n$
    برابر جمع ضرایبی است که به کل کسر اعمال می‌شود یعنی
    $\sum_{i = 0}^{n} \prod_{j = 0 \neq i}^{n} \frac{1}{(x_i - x_j)} f(x_i)$.
    به عبارت دیگر در عبارت درون‌یابی شده، پشت 
    $x^n$
    عدد فوق واقع است. از طرفی می‌دانیم
    جواب
    \LR{interpolation}
    در این دو روش یکسان (یونیک) است، پس ضریب 
    $x^n$
    در
    \LR{Divided Difference}
    که برابر با
    $f[x_0, \dots, x_n]$
    است، با ضریب 
    $x^n$
    در روش لاگرانژ باید برابر باشد وگرنه چندجمله‌ای یکی نخواهد بود. در نهایت داریم
    \begin{align*}
        f[x_0, \dots, x_n] = \sum_{i = 0}^{n} \prod_{j = 0 \neq i}^{n} \frac{1}{(x_i - x_j)} f(x_i) = \sum_{i = 0}^{n} \frac{f(x_i)}{\prod_{j = 0 \neq i}^{n} (x_i - x_j)}
    \end{align*}
    \item 
    از رابطه
    \ref{eq:1}
    بخش قبل اقدام می‌کنیم
    (روش 
    \LR{Divided Difference}).
    تابع 
    اینترپولیشن روی نقاط
    $x_0$
    تا
    $x_n$
    است، پس در این نقاط تابع
    $f$
    اصلی با اینترپولیشن خود (تعریف کنیم برابر با
    $P_n(x)$)
    برابر است. پس تابع
    $P_n(x) - f(x)$,
    $n + 1$
    ریشه دارد. می‌دانیم بین هر دو ریشه چنین تابعی که مشتق‌پذیر و پیوسته است، مشتق صفر است.
    پس اگر در 
    $n + 1$
    نقطه ریشه داشته باشد، بین هر دوتا در یکی مشتق صفر است پس در کل در 
    $n$
    نقطه مشتق صفر است. مشابها از روی همین 
    $(P_n(x) - f(x))^{'}$
    اگر مشتق بگیریم، در 
    $n - 1$
    نقطه مشتق آن صفر است، و به همین شکل مشتق 
    $n$
    ام 
    $P_n(x) - f(x)$
    در یک نقطه که آن را همان
    $\xi$
    فرض می‌کنیم صفر است (مشخصا 
    $\xi$
    در بین این نقاط است زیرا در هر مرحله جایی که مشتق صفر می‌شد بین همان نقاط قبلی بود). در این نقطه‌ی 
    $\xi$،
    مقدار مشتق 
    $n$
    ام 
    $f$
    و
    $P_n$
    برابر است. یعنی
    \begin{align*}
        f^{(n)}(\xi) =\overbrace{n(n-1) \dots (1)}^{n!} f[x_0, \dots,x_n] \xRightarrow{(n! \neq 0)} f[x_0, \dots ,x_n] = \frac{f^{(n)}(\xi)}{n!}
    \end{align*}
    در حل این سوال از تمرین‌های قرار داده شده در سایت درس از سال‌های قبل آموخته شده است.
\end{enumerate}
}

