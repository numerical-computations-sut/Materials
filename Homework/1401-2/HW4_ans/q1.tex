\\
جدول مقادیر زیر را در نظر بگیرید:

\begin{latin}
\begin{table}[H]
  \begin{center}
    \begin{tabular}{c|c c c}
      \textbf{$x$} & $0$ & $1$ & $2$ \\
      \hline
      \textbf{$f(x)$} & 9.90 & 7.94 & 23.00 \\
    \end{tabular}
  \end{center}
\end{table}
\end{latin}

\begin{enumerate}
	\item
	ابتدا چندجمله‌ای درون‌یاب
	$P_2(x)$
	را برای داده‌های بالا محاسبه کنید،
	سپس از این چندجمله‌ای مشتق بگیرید.
	
	مقدار $x$
	برابر 0
	را در $P'_2(x)$
	قرار دهید و به این ترتیب $f'(0)$
	را تخمین بزنید.
	
	\item
	مقدار
	$f'(0)$
	را با استفاده از فرمولی که خطای آن از
	$O(h^2)$
	است تخمین بزنید.
	
	\item
	مفدار واقعی $f'(0)$
	برابر
	$-3.5196$
	است.
	خطای دو روش محاسبه مشتق را با یک‌دیگر مقایسه کنید.
	
	\item
	تعداد عملیات‌های ضرب، تقسیم، جمع و تفریقی که هرکدام از این دو روش به آن نیازمند بودند را با هم مقایسه کنید.
\end{enumerate}
\textcolor{blue}{حل
\begin{enumerate}
    \item 
    \begin{align*}
    \begin{tabular}{cccc}
        \text{مرتبه دوم} & \text{مرتبه اول} & $f(x)$ & $x$ \\
        & & $9.9$ & $0$ \\
       & $-1.96$ & &\\
     $8.51$ & & $7.94$ & $1$
       & & $15.06$ & \\
        & & $23$ & $2$
    \end{tabular}
    \end{align*}
    \begin{align*}
        &P_2(x) = 9.9 - 1.96 (x - 0) + 8,51 (x - 0) (x - 1) \\
        \Rightarrow &P^{'}_2(x) = - 1.96 + 2 (8.51)x - 8.51 \\
        \Rightarrow &f^{'}(0) \approx P^{'}_2(0) = -1.96 - 8.51 = -10.47
    \end{align*}
    \item 
    \begin{align*}
        f^{'}(x_0) &\approx \frac{1}{2h} (-3 f(x_0) + 4f(x_0 + h) - 3 f(x_0 + 2h)) \\
        &= \frac{1}{2 (1)} (-3 (9.9 + 4 (7.94) - 23) = -10.47
    \end{align*}
    \item 
    هر دو روش به عدد یکسان 
    $-10.47$
    رسیدند پس خطای یکسانی دارند:
    $|-3.5196 - (-10.47)|$
    \item 
    \begin{itemize}
        \item روش اول: حداقل ۱۰ محاسبه
        \item روش دوم ۶ محاسبه
    \end{itemize}
    با توجه به نتایج بالا مشخص است روش دوم بهتر است.
\end{enumerate}
}

