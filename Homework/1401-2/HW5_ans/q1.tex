معادله دیفرانسیل
$y' = \lambda y$
را در نظر بگیرید.
\begin{enumerate}
    \item 
    نشان دهید روش اویلر برای این معادله به ازای
    $\lambda < 0$ و مقدار ابتدایی 
    $y(0) = 1$
    برای طول گام
    
    $-z < \lambda h < 0$
    پایدار است; یعنی
    $\lim_{{n \to \infty}} y_n = 0$.
    \item 
    نشان دهید در حل معادله بالا می‌توان روش رانگ - کوتای مرتبه چهارم را به صورت زیر نوشت:
    \[y_{i+1} = (1 + h\lambda + \frac{1}{2}(h\lambda)^2 + \frac{1}{6}(h\lambda)^3 + \frac{1}{24}(h\lambda)^4)y_i\]
\end{enumerate}
\textcolor{blue}{حل
\begin{enumerate}
    \item 
    \begin{align*}
        y_1 &= y_0 + h\underbrace{(\lambda y_0)}_{f(x_0,y_0)} = y_0 (\lambda h + 1) \\
        y_2 &= y_1 + h \underbrace{(\lambda y_1)}_{f(x_1,y_1)} = y_1 (\lambda h + 1) = y_0 (\lambda h + 1)^2 \\
        &\vdots \\
        y_n &= y_{n - 1} + h\underbrace{(\lambda y_{n - 1})}_{f(x_{n - 1}, y_{n - 1})} = y_{n - 1}(\lambda h + 1) = y_0 (\lambda h + 1)^n
    \end{align*}
    \begin{align*}
        -2 < \lambda h < 0 \Rightarrow -1 < \lambda h + 1 < 1
    \end{align*}
    و عددی بین 
    -۱ و 
    ۱ اگر به توان
    $n \rightarrow \infty$
    برسد، عملا صفر می‌شود. پس
    \begin{align*}
        \lim_{n \to \infty} y_n = \lim_{n \to \infty} y_0 (\lambda h  + 1)^n= 0
    \end{align*}
    \item 
    می‌دانیم در رانگ‌کوتای مرتبه ۴:
    \begin{align*}
        k_1 &= hf(x_n,y_n) = h \lambda y_n \\
        k_2 &= hf(x_n + \frac{h}{2}, y_n + \frac{k_1}{2}) = h \lambda (y_n + \frac{h \lambda y_n}{2}) = h \lambda y_n (\frac{h \lambda}{2} + 1) \\
        k_3 &= hf(x_n + \frac{h}{2}, y_{n} + \frac{k_2}{2}) = h\lambda (y_n + \frac{k_2}{2}) = h \lambda (y_n + \frac{h \lambda y_n}{2} (\frac{h \lambda }{2} + 1)) = h \lambda y_n (1 + \frac{h^2 \lambda^2}{4} + \frac{h \lambda}{2}) \\
        k_4 &= hf(x_n + h, y_n + k_3) \\
        k_4 &= h \lambda (y_n + k_3) = h \lambda (y_n + h \lambda y_n (1 + \frac{h^2 \lambda^2}{4} + \frac{h \lambda}{2})) = h \lambda y_n (1 + h \lambda + \frac{h^3 \lambda^3}{4} + \frac{h^2 \lambda^2}{2}) \\
        y_{n + 1} &= y_{n} + \frac{k_1 + 2k_2 + 2k_3 + k_4}{6} \\
        \Rightarrow y_{n + 1} &= y_{n} + h \lambda y_{n} \frac{1 + 2(\frac{h \lambda}{2} + 1) + 2(1 + \frac{h^2 \lambda^2}{4} + \frac{h \lambda}{2}) + 1 + h \lambda + \frac{h^3 \lambda^3}{4} + \frac{h^2 \lambda^2}{2}}{6}
    \end{align*}
    حال با جایگذاری $i$
    به جای 
    $n$
    و سپس فاکتورگیری از 
    $y_i$
    خواهیم داشت:
    \begin{align*}
        y_{i + 1} &= y_i (1 + h\lambda \frac{1 + h\lambda + 2 + 2 + \frac{h^2 \lambda^2}{2} + h\lambda + 1 + h\lambda + \frac{h^3 \lambda^3}{4} + \frac{h^2 \lambda^2}{2}}{6}) \\
        &= y_i (1 + (h \lambda) \frac{6}{6} + (h \lambda) (\frac{3 h \lambda}{6}) + (h\lambda)(\frac{h^2 \lambda^2}{6}) + (h\lambda) (\frac{h^3 \lambda^3}{4 \times 6})) \\
        &= y_i (1 + h\lambda + \frac{(h \lambda)^2}{2} + \frac{(h \lambda)^3}{6} +\frac{(h\lambda)^4}{24})
    \end{align*}
    که همان صورت سوال است.
\end{enumerate}
}

