\\
معادله‌ی خطی
$Ax = b$
را با مقادیر زیر درنظر بگیرید:
\begin{LTR}
\begin{center}
A = $\begin{pmatrix}
4 & 1 & -2\\
-1 & 4 & -1\\
1 & -1 & 4
\end{pmatrix}$ و
b =  $\begin{pmatrix}
4\\
0\\
4
\end{pmatrix}$
\end{center}
\end{LTR}

\begin{enumerate}
\item
با تقریب اولیه‌ی
$x^{(0)} = 0$
و سه تکرار از روش ژاکوبی
\footnote{Jacobi}
مقادیر
$x^{(1)}$
،
$x^{(2)}$
و
$x^{(3)}$
را محاسبه کنید.

\item
با تقریب اولیه‌ی
$x^{(0)} = 0$
و سه تکرار از روش گاوس-سایدل
\footnote{\lr{Gauss-Seidel}}
مقادیر
$x^{(1)}$
،
$x^{(2)}$
و
$x^{(3)}$
را محاسبه کنید.
\item
کدام یک از روش‌های بالا تقریب بهتری از جواب‌ها می‌دهد؟

\end{enumerate}

\textcolor{blue}{حل
\\
\begin{enumerate}
    \item دور اول:
    \begin{align*}
        \begin{cases}
        4x_1 + 0 - 2 (0) = 4 \\
         -(0) + 4x_2 - (0) = 0 \\
        0 - (0) + 4x_3 = 4
        \end{cases}
        \Rightarrow 
        \begin{cases}
            x_1 = 1 \\
            x_2 = 0 \\
            x_3 = 1
        \end{cases}
    \end{align*}
    دور دوم:
    \begin{align*}
        \begin{cases}
            4x_1 + 0 - 2(1) = 4 \\
            -1 + 4x_2 - 1 = 0 \\
            1 - 0 + 4x_3 = 4
        \end{cases} \Rightarrow 
        \begin{cases}
            x_1 = 1.5 \\
            x_2 = 0.5 \\
            x_3 = 0.75
        \end{cases}
    \end{align*}
    دور سوم:
    \begin{align*}
        \begin{cases}
            4x_1 + 0.5 - 2(0.75) = 4 \\
            -1.5 + 4x_2 - 0.75 = 0 \\
            1.5 - 0.5 + 4x_3 = 4
        \end{cases}
        \Rightarrow \begin{cases}
            x_1 = 1.25 \\
            x_2 = 0.5625 \\
            x_3 = 0.75
        \end{cases}
    \end{align*}
    \item دور اول:
    \begin{align*}
        &4x_1 + 1 (0) - 2 (0) = 4 \rightarrow x_1 = 1 \\
        &-1 (x_1) + 4x_2 - 1(0) = 0 \rightarrow 4x_2 = x_1 \xrightarrow{x_1 = 1} x_2 = 0.25 \\
        &x_1 - 1(x_2) + 4x_3 = 4 \xrightarrow{(x_1, x_2) = (1,0.25)} 1 - 0.25 + 4x_3 = 4 \rightarrow x_3 = 0.8125
    \end{align*}
    دور دوم:
    \begin{align*}
        &4x_1 + 0.25 - 2(0.8125) = 4 \rightarrow x_1 = 1.34375 \\
        &-x_1 + 4x_2 - 0.8125 = 0 \xrightarrow{\text{از بالایی}} x_2 = 0.5390625 \\
        &x_1 - x_2 + 4x_3 = 4 \xrightarrow{\text{از بالایی ها}} x_3 = 0.798828125
    \end{align*}
    \newpage
    دور سوم:
    \begin{align*}
        &4x_1 + 0.5390625 - 2(0.798828125) = 4 \rightarrow x_1 = 1.2646484375 \\
        &-x_1 + 4x_2 - 0.798828125 = 0 \xrightarrow{\text{از بالایی}} x_2 = 0.5158691406 \\
        &x_1 - x_2 + 4x_3 = 4 \xrightarrow{\text{از بالایی ها}} x_3 = 0.8128051758
    \end{align*}
    \item 
    جواب‌ واقعی (بدست امده از ولفرام)
    \begin{align*}
        \begin{cases}
            x_1 = 1.2753623188 \\ 
            x_2 = 0.5217391304 \\
            x_3 = 0.8115942029
        \end{cases}
    \end{align*}
    حال اختلاف‌ جواب‌های روش‌های بخش الف و ب را مقایسه می‌کنیم
    \\
    اختلاف روش گاوس سایدل:
    \begin{align*}
        \begin{cases}
            \Delta x_1^{A} = 0.0107138743 \\
            \Delta x_2^{A} = 0.0058699898 \\
            \Delta x_3^{A} = 0.0012109729
        \end{cases}
    \end{align*}
    اختلاف روش ژاکوبی:
    \begin{align*}
        \begin{cases}
            \Delta x_1^{B} = 0.0253623188 > \Delta x_1^{A} \\
            \Delta x_2^{B} = 0.0407608696 > \Delta x_2^{A} \\
            \Delta x_3^{B} = 0.0615942029 > \Delta x_3^{A}
        \end{cases}
    \end{align*}
    به وضوح جواب‌های بدست آمده از (ب) نزدیکتر هستند، پس گاوس سایدل تقریب بهتری از جواب‌ها می‌دهد.
\end{enumerate}
}