\begin{enumerate}
    \item نشان دهید روش گاوس-سیدل 
    در مورد دستگاه
\begin{center}
    \begin{cases}
    5x_1 + 3x_2 + 4x_3 = 12 \\
    3x_1 + 6x_2 + 4x_3 = 13 \\
    4x_1 + 4x_2 + 5x_3 = 13
    \end{cases}
\end{center}
برای بردار اولیه 
$x^{(0)} = 0$
همگراست، در حالی که روش ژاکوبی واگراست.
\item 
آیا ماتریس ضرایب دستگاه فوق غالب قطری است؟
\item 
آیا شرط غالب قطری بودن ماتریس ضرایب در یک دستگاه شرط لازم است؟
\end{enumerate}

\textcolor{blue}{
حل
\\
\begin{enumerate}
\item
مقادیر ویژه‌ی ماتریس
$A = \begin{pmatrix}
    5 & 3 & 4 \\
    3 & 6 & 4 \\
    4 & 4 & 5
\end{pmatrix}$
برابرند با
$
\begin{cases}
    \lambda_1 \approx 12.6873 \\
    \lambda_2 \approx 2.5356 \\
    \lambda_3 \approx 0.7771
\end{cases}
$
\\ \\
که هر سه مثبت‌اند ضمنا
ماتریس متقارن است
$(A^T = A)$.
در نتیجه مثبت معین 
(\lr{Positive definite})
است و در نتیجه در روش گاوس سیدل همگرا است. اما روش 
ژاکوبی واگراست و با اجرای کد نیز به این نتیجه می‌رسیم که برخلاف گاوس سایدل جواب درستی نمی‌دهد. (شروط مثبت معین بودن برای گاوس سیدل است نه ژاکوبی و چون قطری غالب هم نیست امکان
\lr{diverge}
هم طبعا دارد)
\\
ضمنا برای اثبات واگرا بودن در ژاکوبی می‌توان مقادیر ویژه
$D^{-1} (L + U)$
را حساب کرد:
\begin{align*}
    \begin{pmatrix}
        5 & 3 & 4 \\
        3 & 6 & 4 \\
        4 & 4 & 5
    \end{pmatrix} = L +D + V = 
    \begin{pmatrix}
        0 & 0 & 0 \\
        3 & 0 & 0 \\
        4 & 4 & 0
    \end{pmatrix} + 
    \begin{pmatrix}
        5 & 0 & 0 \\
        0 & 6 & 0 \\
        0 & 0 & 5
    \end{pmatrix} +
    \begin{pmatrix}
        0 & 3 & 4 \\
        0 & 0 & 4 \\
        0 & 0 & 0
    \end{pmatrix}
\end{align*}
\begin{align*}
    \Rightarrow D^{-1} (L + U) = \begin{pmatrix}
        0 & \frac{3}{5} & \frac{4}{5} \\
        \frac{1}{2} & 0 & \frac{2}{3} \\
        \frac{4}{5} & \frac{4}{5} & 0
    \end{pmatrix} \xrightarrow{\text{مقادیر ویژه}}  (1.39054, -0.847394, -0.543141)
\end{align*}
ماکسیمم قدرمطلق‌شان 
$1.39054$
است که 
$1 < $
است. در نتیجه واگرا می‌شود.
\\
اما برای گاوس سایدل
$(L + D)^{-1} U$
را حساب می‌کنیم:
\begin{align*}
    \begin{pmatrix}
        0 & \frac{3}{5} & \frac{4}{5} \\
        0 & -\frac{3}{10} & \frac{4}{15} \\
        0 & -\frac{6}{25} & -\frac{64}{75}
    \end{pmatrix} \xrightarrow{\text{مقادیر ویژه}} (0,
    -0.688669, 
    -0.464665)
\end{align*}
که ماکسیمم قدرمطلق‌شان 
$1 > 0.688669$
است. که با همگرا بودن گاوس سایدل سازگار است.
\\
پس گاوس سایدل همگرا و ژاکوبی واگراست.
\item 
خیر زیرا به عنوان مثال در سطر اول
$5 \ngtr 3 + 4$
است.
(
و
$6 \ngtr 3 + 4$
و
$5 \ngtr 4 + 4$)
\item 
خیر زیرا اگر شرط لازم بود، چون 
$A$
در گاوس سایدل همگرا بود، باید ماتریس غالب قطری می‌شد، ولی نشده است. در نتیجه شرط لازم نمی‌تواند باشد. (در حقیقت غالب قطری بودن یا متقارن مثبت معین بودن شرط کافی است)
\end{enumerate}
}