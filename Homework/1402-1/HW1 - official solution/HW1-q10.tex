فرض کنید می‌خواهیم تابع 
$f(x)$
را با استفاده از سری مک‌لورن توابع $sin(x)$، $cos(x)$ و $tan^{-1}(x)$ روی بازه‌ی $[\frac{\pi}{2}, \frac{5\pi}{6}]$ حساب کنیم.
محاسبه کنید که هر کدام از این سه سری را حداقل باید تا چه مرتبه ای بسط دهیم تا خطایِ کلِ محاسبه‌یِ مقدارِ تابع روی کل بازه، از $10^{-5}$ کمتر شود.

$$f(x) = \frac{ 4sin^2(x) - cos(x) }{3tan^{-1}(\frac{2x}{\pi \sqrt{3}})}$$

\begin{comment}
\end{comment}
پاسخ:

ابتدا تابع k را به صورت زیر تعریف می‌کنیم:
$$k(a,b,c) = \frac{4a^2 - b}{3c}$$
حال مشاهده می‌کنیم که 
$$f(x) = k(sin(x), cos(x), tan^{-1}(\frac{2x}{\pi \sqrt{3}}))$$
حال با استفاده از فرمول خطای چند متغیر داریم:
$$\Delta k = \frac{\partial k}{\partial a} \Delta a + \frac{\partial k}{\partial b} \Delta b + \frac{\partial k}{\partial c} \Delta c$$

$$\frac{\partial k}{\partial a} = \frac{8a}{3c}$$
$$\frac{\partial k}{\partial b} = \frac{-b}{3c}$$
$$\frac{\partial k}{\partial c} = \frac{-4a + b}{3c^2}$$

حال حد بالایی برای مقادیر ماکسیمم این ضریب خطاها را روی بازه‌ی داده شده می‌یابیم:
برای سادگی ماکسیمم سینوس و کسینوس را ۱ درنظر می‌گیریم. همچنین دقت می‌کنیم که تابع وارون تانژانت صعودی است پس مقدار کمینه آن روی بازه، در نقطه $x=\pi/2$ اتفاق می‌افتد.

$$\abs{\frac{\partial k}{\partial a}} \leq \frac{8}{3 (\frac{\pi}{6})}$$
$$\abs{\frac{\partial k}{\partial b}} \leq \frac{1}{3 (\frac{\pi}{6})}$$
$$\abs{\frac{\partial k}{\partial c}} \leq \frac{5}{3 (\frac{\pi}{6})^2}$$

حال داریم:

$$|\Delta k| \leq \abs{\frac{\partial k}{\partial a} \Delta a} + \abs{\frac{\partial k}{\partial b} \Delta b} + \abs{\frac{\partial k}{\partial c} \Delta c}$$

$$|\Delta k| \leq \abs{ \frac{8}{3 (\frac{\pi}{6})} \Delta a} + \abs{\frac{1}{3 (\frac{\pi}{6})} \Delta b} + \abs{\frac{5}{3 (\frac{\pi}{6})^2} \Delta c}$$

حال داریم:

$$\frac{8}{3 (\frac{\pi}{6})} \approx 5.092$$
$$\frac{1}{3 (\frac{\pi}{6})} \approx 0.636$$
$$\frac{5}{3 (\frac{\pi}{6})^2} \approx 6.079$$

و میبینیم که 

$$10^{-2} \times 5.092 + 10^{-1} \times 0.636 + 10^{-1} \times 6.079 \approx 1$$

درواقع برای اینکه تضمین کنیم که خطای کل عبارت از $10^{-5}$ کمتر است. باید حتما خطای هر سه جمله کمتر از $10^{-6}$ باشد. اما خطای یکی از جملات باید کمتر از $10^{-7}$ باشد.

در حقیقت هر ترکیبی از خطاها که معادله زیر را ارضا کند می‌تواند جواب مساله باشد و بسته به توان محاسبانی می‌توانیم هر کدام را انتخاب کنیم.

$$5.092 \Delta a + 0.636 \Delta b + 6.079 \Delta c  \leq 10^{-5}$$

برای مثال حالت زیر را انتخاب می‌کنیم.
$$\Delta a  = 10^{-7}$$
$$\Delta b = 10^{-6}$$
$$\Delta c = 10^{-6}$$

و با استفاده از حد بالای خطای سری هر کدام تعداد جملات مورد نیاز را حساب می‌کنیم:

برای sin(.) :
$$|\Delta sin(x)| \leq \abs{\frac{x^{2n+1}}{(2n+1)!}} \leq 10^{-7}$$
$$\frac{(\frac{5\pi}{6})^{2n+1}}{(\frac{2n+1}{e})^{2n+1}} \leq 10^{-7}$$
$$(\frac{5e\pi}{6(2n+1)})^{2n+1} \leq 10^{-7}$$
$$n=9 \Rightarrow (\frac{5 \times 8.54}{6 \times 19})^{19} < (\frac{2}{5})^{19} < 10^{-7}$$

برای cos(.) : 
$$|\Delta cos(x)| \leq \abs{\frac{x^{2n}}{(2n)!}} \leq 10^{-6}$$
$$\frac{(\frac{5\pi}{6})^{2n}}{(\frac{2n}{e})^{2n}} \leq 10^{-6}$$
$$(\frac{5e\pi}{2n})^{6(2n)} \leq 10^{-6}$$
$$n=9 \Rightarrow (\frac{5 \times 8.54}{6 \times 18})^{18} < (\frac{2}{5})^{18} < 10^{-6}$$

برای $tan^{-1}(.)$ :
$$|\Delta tan^{-1}(\frac{2x}{\pi \sqrt{3}})| \leq \abs{\frac{(\frac{5}{3\sqrt{3}})^{2n+3}}{(2n+3)}} \leq 10^{-6}$$
$$\frac{(0.962)^{2n+3}}{2n+3} \leq 10^{-6}$$
با لگاریتم گیری بدست می‌آید
$$n=108$$

به دلیل عدم وجود فاکتوریل در مخرج جملات این سری، سری بسیار کند همگرا می‌شود. این کندی در تعداد عبارات مورد نیاز برای بدست آوردن دقت مورد نیاز کاملا مشهود است.

