بسط تیلور تابع 
$tan^{-1} (x)$
را  حول 
$x_0 = 0$
بیابید. شما باید جمله کلی برای 
$P_n(x)$
پیدا کنید، بازه همگرایی آن را مشخص کنید و حد بالای مناسبی برای 
$|R_n(x)|$
برحسب n ارائه دهید. (فرض کنید که می‌دانید که این تابع روی اعداد حقیقی تحلیلی است)

\begin{comment}
\end{comment}
پاسخ:

ابتدا بست مک‌لورن تابع را حساب می‌کنیم.
برای بدست آوردن بست مک‌لورن وارون تانژانت، دو روش قابل قبول وجود دارد. یا میتوانید از تابع داده شده، متوالیا مشتق بگیرید و به کمک استقرا سری کامل را بسازید.

راه دیگر این است که از بسط مک‌لورن تابع $\frac{1}{1-x}$ استفاده کنید:

$$-1 < x < 1$$
$$\frac{1}{1-x} = \sum_{k=0}^{\infty} x^k$$
$$\frac{1}{1+x^2} = \sum_{k=0}^{\infty} (-x^2) ^k = \sum_{k=0}^{n} (-1)^k x^{2k}$$
$$\int_{0}^{x} \frac{1}{1+t^2}dt = tan^{-1}(x)$$
$$\int_{0}^{x} \frac{1}{1+t^2}dt = \int_{0}^{x} \sum_{k=0}^{\infty} (-1)^k t^{2k} dt$$
$$tan^{-1}(x) = \sum_{k=0}^{\infty} (-1)^k \int_{0}^{x} t^{2k} dt$$
$$tan^{-1}(x) = \sum_{k=0}^{\infty} (-1)^k \frac{x^{2k+1}}{2k+1}$$

حال با دانستن این که تابع تحلیلی است، طبق قضیه باقی‌مانده تیلور میدانیم که بسط تیلور این تابع حول صفر وجود دارد به طوری که 

$$P_n(x) = \sum_{k=0}^{n} \frac{f^{(k)}(0)}{k!} x^k$$
$$R_n(x) = \frac{f^{(n+1)}(c)}{(k+1)!} x^{n+1}$$

از بسط مکلورن به دست آمده در بالا به این نتیجه می‌رسیم که
\[
f^{(k)}(0) = 
                \begin{cases}
                    \text{0} &\quad \text{k is even} \cr
                    (-1)^{\frac{k-1}{2}} (k-1)! &\quad \text{k is odd}
                \end{cases}
\]

فرم کلی $P_n$ به صورت canonical

$$P_n(x) = \sum_{k=0}^{n} \frac{f^{(k)}(0)}{k!}x^k$$

فرم کلی $P_n(x)$ به صورت کوتاه

$$P_{2n+1}(x) = \sum_{k=0}^{n} (-1)^k \frac{x^{2k+1}}{2k+1}$$

هر کدام از این دو صورت را نوشته باشید قبول است.

بازه همگرایی تابع نمیتواند از $(-1,1)$ بیشتر باشد زیرا ما از بسط مکلورن $\frac{1}{1-x}$ استفاده کردیم که در این بازه همگرا است.
همچنین با توجه به بسط مکلورن به دست آمده برای $tan^{-1}(x)$ میتوان دید که سری alternating هست و در صورتی که x در بازه $(-1,1)$ باشد، آنگاه تک تک جملات در حد بی نهایت k به صفر میل می‌کنند و قدر مطلق هر جمله از جمله قبلی کوچکتر است پس این سری در این بازه مطلفا همگراست. از عکس این مشاهده می‌توان واگرا بودن سری در خارج بازه را هم نتیجه گرفت.

برای بدست آوردن حد بالای خطای این سری. به این موضوع دقت می‌کنیم که این دنباله، همگرا و  alternating است و به همین‌خاطر خطای آن همیشه کمتر از جمله ی بعدی دنباله است

$$|R_{2n+1}(x)| \leq |P_{2n+3}(x) - P_{2n+1}(x)| = \abs{\frac{x^{2n+3}}{2n+3}}$$

