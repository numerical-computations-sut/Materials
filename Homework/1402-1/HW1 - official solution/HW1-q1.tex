فرمول سود یک شرکت برابر است با 
$P=\frac{5}{6}t^{3}- et^{2}$
که 
t
در اینجا زمان به میلی‌ثانیه است. خطاهای محتمل از انواع خطای مدل، خطای اندازه گیری، خطای گرد کردن و خطای عملیات را در این آزمایش به طور مختصر توضیح دهید.

\begin{comment}

پاسخ:
\begin{itemize}
    \item خطای مدل: اینکه واقعا سود و زیان از این فرمول پیروی می‌کند یا نه.
    \item خطای ضرب و تفریق(خطای عملیات): چون اعداد در این مثال اعشاری هیتند و این اعشار تا بی‌نهایت ادامه دارد، هنگام ضرب کردن و یا تفریق کردن، ممکن است با خطای ضرب یا تفریق مواجه شویم.
    \item خطای اندازه‌گیری: اگر مقدار t را درست حساب نکیم، مرتکب این نوع خطا شده‌ایم که در واقع ناشی از خطای وسیله و شخص ‌اندازه‌گیر است.
    \item خطای گردکردن: وقتی می‌خوایم $e$ را تقریب بزنیم باید آن را گرد کنیم. این گرد کردن یعنی ایجاد فاصله با مقدار واقعی آن.
\end{itemize}

\end{comment}