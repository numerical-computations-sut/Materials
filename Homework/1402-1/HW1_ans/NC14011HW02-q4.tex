چند جمله‌ای $f(x) = x^5 + x^3 + x^2 + x - 4.4$ را در نظر بگیرید. این چند جمله‌ای را یک بار از طریق گرد کردن و یک بار دیگر از طریق قطع کردن تا سه رقم اعشار به ازای $x = 2.66$ بدست آورید و خطای نسبی را حساب کنید. چگونه میتوان خطای تابع را کم کرد؟ سپس این تابع را به گونه‌ای تغییر دهید تا خطا در قسمت قبل کمتر شود.\\


پاسخ:\\
$$f(x) = 157.3272428576$$
$$f_{round}(x) = 157.328$$
$$f_{truncated}(x) = 157.326$$


برای کم کردن خطا باید تعداد ضرب‌ها را کم تر کنیم. بنابراین تایع را به صورت زیر بازنویسی می‌کنیم:\\

$$f(x) = x (1 + x (1 + x (1 + x^2))) - 4.4$$
$$f_{round}(x) = 157.333$$
$$f_{truncated}(x) = 157.312$$