مقادیر تابع 
\(f\)
به صورت جدول زیر در دست است:
\begin{center}
	\begin{tabular}{ccccc|c}
		& \(3\) &\(1\)&\(0\) & \(-1\) & \(x_i\)\\
		\hline
		& \(37\) & \(9\) & \(2\) &\(0\) & \(f(x_i)\)
	\end{tabular}
\end{center}
معادلات اسپلاین مکعبی طبیعی  که تابع 
\(f\)
را در نقاط
\(x_0=-1\)،
\(x_1=0\)،
\(x_2=1\) 
و  
\(x_3=3\)
درونیابی نماید را به‌دست آورید.\\
توجه کنید منظور از اسپلاین مکعبی طبیعی در نظر گرفتن شرط
\(s''(x_0) = s''(x_n) = 0\)
است که در آن
\(s(x)\)
تابع اسپلاین  است. 