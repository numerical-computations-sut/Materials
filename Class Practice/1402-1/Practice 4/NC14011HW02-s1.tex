\\
محاسبه مشتق با استفاده از فرمول تیلور
\footnote{Taylor}
:

فرض کنید تابع \( f \) روی \([a, b]\) به قدر لازم مشتق‌پذیر است و \( i = 0, 1, \ldots, n , x_i \)،  نقاط هم‌فاصله در این بازه باشند با قیود \( a = x_0 \) و \( b = x_n \). بنابراین \( x_{i+1} = x_i + h \) به ازای \( h>0 \). طبق قضیه تیلور داریم:

\[ f(x_{i + 1}) = f(x_i + h) = f(x_i) + hf'(x_i) + \frac{h^2}{2!}f''(x_i) + \frac{h^3}{3!}f'''(\xi) , \quad x_i < \xi < x_{i+1} \]

بنابراین

\[ f(x_{i + 1}) = f(x_i) + hf'(x_i) + \frac{h^2}{2!}f''(x_i) + O(h^3) \]

به طریق مشابه

\[ f(x_{i-1}) = f(x_i) - hf'(x_i) + \frac{h^2}{2!}f''(x_i) + O(h^3) \]

در نتیجه

\[ f'(x_i) = \frac{f(x_{i+1}) - f(x_{i-1})}{2h} + O(h^2) \]

یعنی \( \frac{f_{i+1} - f_{i-1}}{2h} \) تخمینی برای \( f'(x_i) \) با خطای برشی \( O(h^2) \) است.